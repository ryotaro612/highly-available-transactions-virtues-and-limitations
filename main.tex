% unicodeは、hyperrefへの指定で、pdfのメタデータにあるタイトルの文字化けを防ぐ
% ptは細かい指定はできないらしい
% A cheat sheet
% https://www.cpt.univ-mrs.fr/~masson/latex/Beamer-appearance-cheat-sheet.pdf
\documentclass[unicode, 14pt, aspectratio=169]{beamer}
\usetheme{titech}
\date{\today}
\title{一貫性と可用性の分類学}
\author{\texttt{ryotaro612}}
\newcommand\blfootnote[1]{%
  \begingroup
  \renewcommand\thefootnote{}\footnote{#1}%
  \addtocounter{footnote}{-1}%
  \endgroup
}

\begin{document}
\begin{frame}[noframenumbering, plain]
\titlepage
\end{frame}
\section{Jepsen}
\begin{frame}[t]
  \frametitle{Jepsen: 分散システムのテストフレームワーク}
  {\large Jepsenはフォールトインジェクションを効率化する}
\end{frame}
\begin{frame}[t]
  \frametitle{Jepsenの高い認知}
  {\large 今日までに45のストレージがテストされた}
  \par
  テスト対象側のetcdからもJepsenに言及がある
\end{frame}
\begin{frame}[t]
  \frametitle{テスト対象を評価するときの課題}
  {\large 一貫性と可用性のありえるペアを決めること}
  % capが粒度が小さいことを意味しよう
\end{frame}
\section{一貫性と可用性のパターン}
\begin{frame}[t]
  \frametitle{a}
  {\large 一貫性と可用性のありえるペアを決めること}
  % capが粒度が小さいことを意味しよう
\end{frame}
\end{document}
