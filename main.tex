% unicodeは、hyperrefへの指定で、pdfのメタデータにあるタイトルの文字化けを防ぐ
% ptは細かい指定はできないらしい
% A cheat sheet
% https://www.cpt.univ-mrs.fr/~masson/latex/Beamer-appearance-cheat-sheet.pdf
\documentclass[unicode, 14pt, aspectratio=169]{beamer}
\usetheme{titech}
\addbibresource{main.bib}
\date{\today}
\title{一貫性と可用性によるシステムの分類}
\author{\texttt{ryotaro612}}
\newcommand\blfootnote[1]{%
  \begingroup
  \renewcommand\thefootnote{}\footnote{#1}%
  \addtocounter{footnote}{-1}%
  \endgroup
}

\begin{document}
\begin{frame}[noframenumbering, plain]
\titlepage
\end{frame}
\section{Jepsen}
\begin{frame}[t]
  \frametitle{Jepsen: 分散システムのテストフレームワーク\cite{jepsen}}
  {\large Jepsenはフォールトインジェクションに特化}
  \begin{itemize}
  \item Clojureのライブラリ
  \item テストケースは次のProtocolの実装
    \begin{itemize}
    \item 分散システムのクライアント
    \item 起こしたい障害
    \item クライアントの操作と障害のスケジュール
    \item 実行結果の確認
    \end{itemize}
  \item Jepsenに実装を呼出、実行結果の保存、再現をまかせる
  \end{itemize}
\end{frame}
\begin{frame}[t]
  \frametitle{Jepsenの高い認知}
  {\large 今日までに45のストレージがテストされた}
  \par
  テスト対象側のetcdからもJepsenに言及がある
\end{frame}
\begin{frame}[t]
  \frametitle{テスト対象を評価するときの課題}
  {\large 特性の違うシステムを比べ方をきめる}
  % capが粒度が小さいことを意味しよう
\end{frame}
\section{一貫性と可用性のパターン}
\begin{frame}[t]
  \frametitle{トレードオフの一貫性と可用性で比べる}
  {\large 一貫性と可用性のペアは半順序関係になる}
  % capが粒度が小さいことを意味しよう
\end{frame}
\begin{frame}[t]
  \frametitle{\normalsize{Highly Available Transactions Virtues and Limitations\cite{high}}}
  {\large 一貫性と両立できる可用性のペアを半順序関係に整理}
  % capが粒度が小さいことを意味しよう
\end{frame}
\begin{frame}[t]
  \frametitle{2種類の可用性}
  {\large Sticky AvailabilityはHigh availablityの必要条件}
  % capが粒度が小さいことを意味しよう
\end{frame}
\begin{frame}[t]
  \frametitle{くらべられるペアの例}
  {\large WFR, causal, linearizable}
  % capが粒度が小さいことを意味しよう
\end{frame}
\begin{frame}[allowframebreaks,t]
  \frametitle{参考資料}
  \printbibliography
  \nocite{*}
\end{frame}
\end{document}
